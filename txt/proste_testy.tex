\section{Proste testy}

Dla zaprezentowania działania algorytmów SocialPageRank i Adapted PageRank przeprowadzono obliczenia na prostych danych. Wszystkie obliczenia wykonano używając danych podanych w tabeli \ref{fig:test_proste_dane}. 


\begin{figure}[htb]
  \centering
  \begin{tabular}{|c|c|c|c|c|}
    \hline
    \multicolumn{1}{|c|}{}&\multicolumn{2}{c|}{Użytkownicy}\\

    \cline{2-3}
    \multicolumn{1}{|c|}{}&użytkownik 1&użytkownik 2\\
    \hline
 	http://www.ted.com/ & inspiration & \\
	http://www.colourlovers.com/&	design & inspiration \\
	http://www.behance.net/	&portfolio, design & portfolio, inspiration \\
    \hline
  \end{tabular}
  \caption{Dane do prostych testów}
  \label{fig:test_proste_dane}
\end{figure}

\section{Wyniki algorytmu SocialPageRank dla prostych danych}


Dla danych z tabeli \ref{fig:test_proste_dane} macierz $M_{d,u}$ mówiąca o zależności dokumentów z użytkownikami ma postać:

\[
 M_{d,u} =
 \begin{pmatrix}
  1 & 0 \\
  1 & 1 \\
  1 & 2
 \end{pmatrix}
\]

Macierz użytkowników i tagów, $M_{u,t}$:

\[
 M_{u,t} =
 \begin{pmatrix}
  1 & 1 & 1 \\
  2 & 0 & 1 
 \end{pmatrix}
\]

Macierz tagów i dokumentów, $M_{t,d}$:
\[
 M_{t,d} =
 \begin{pmatrix}
  1 & 1 & 1 \\
  0 & 1 & 1 \\
  0 & 0 & 2 
 \end{pmatrix}
\]


\subsection*{wyniki:}
Dla powyższych danych wyniki algorytmu zbiegają po czterech iteracjach z dokładnością
$|P_3 - P_4|  < 10^{-10}$. Wyniki zostały przedstawione w tabelce \ref{fig:social_page_simple_wyniki}


\begin{table}[h]
  \centering
    \begin{tabular}{ | c | c | }
\hline
&Social PageRank \\
\hline
www.ted.com & 0.2381373691295440 \\
www.colourlovers.com  & 0.4343479235414989 \\
www.behance.net & 0.8686958470829979 \\
\hline
\end{tabular}
  \caption{Wyniki działania algorytmu Social PageRank}
  \label{fig:social_page_simple_wyniki}
\end{table}


Można zauważyć, że największy ranking ma strona behance.net. Strona ta została dodana przez dwóch użytkowników. Strona colourlovers.com została dodana przez taką samą liczbę użytkowników, jednak uzyskała ranking o połowę mniejszy. Spowodowane to zostało tym, że behance.net opisana została większą ilością bardziej popularnych tagów: dwa razy tagiem 'portfolio', użytym tylko dla tej strony, raz tagiem 'design' i tagiem 'inspiration', który jest najpopularniejszym tagiem, użytym aż 3 razy. 













\section{Adapted PageRank: wyniki}

\subsection{Wyniki algorytmu dla prostych danych}


Algorytm Adapted PageRank przetestowano na tych samych danych co algorytm Social PageRank (tabelka \ref{fig:test_proste_dane}). Macierz asocjacyjna powstała z tych danych ma wymiary $8 \times 8$ i wygląd:

\[
 G_f =
 \begin{pmatrix}
0 & 0 & 0 & 1	 & 0 & 1 & 0 & 0\\
0 & 0 & 0 & 1 & 1 & 1 & 1 & 0\\
0 & 0 & 0 & 2 & 2 & 1 & 1 & 2\\
1 & 1 & 2 & 0 & 0 & 1 & 2 & 1\\
0 & 1 & 2 & 0 & 0 & 2 & 0 & 1\\
1 & 1 & 1 & 1 & 2 & 0 & 0 & 0\\
0 & 1 & 1 & 2 & 0 & 0 & 0 & 0\\
0 & 0 & 2 & 1 & 1 & 0 & 0 & 0
 \end{pmatrix}
\]

\begin{table}[h]
  \centering
    \begin{tabular}{ | c | c | }
\hline
&Adapted PageRank \\
\hline
doc: http://www.ted.com/ & 0.280676409730572 \\
doc: http://www.colourlovers.com/ & 0.369220441236174 \\
doc: http://www.behance.net/ & 0.384551423972747 \\
\hline
usr: użytkownik A & 0.402473669662513 \\
usr: użytkownik B & 0.354578057974890 \\
\hline
tag: inspiration & 0.383291185401107 \\
tag: design & 0.321455285546253 \\ 
tag: portfolio	 & 0.314739469615726 \\
\hline
\end{tabular}
  \caption{Wyniki działania algorytmu Adapted PageRank}
  \label{fig:adapted_page_simple_wyniki}
\end{table}

Zbierzność wektora została uzyskana po 22 iteracjach. Jest to zdecydowanie dłuższy czas w porównaniu z czterema wymaganymi iteracjami przy poprzenim algorytmie


Analizując wyniki z tabeli \ref{fig:test_proste_dane} można zauważyć, że najwyższy ranking wśród dokumentów ma strona begence.net: została ona dodana przez 2 użytkowników i przypisane jej zostały 4 tagi. Niewiele niższy ranking ma witryna colourlovers dodana przez 2 uzytkowników i opisana 2 różnymi annotacjami. Można po tym wywnioskować ze nadanie większej ilości tagów nie ma dużego wpływu na rank strony. Za to zmniejszenie liczby użytkowników którzy tą strone dodali, ma duze: przykład strona ted.com i colourlovers.com gdzie widoczny jest dość duzy skok wartości wyniku.





\section{Wyniki działania algorytmów}

W tablicach \ref{tab:adapted_wyniki_sorted} i \ref{tab:social_page_wyniki_sorted} przedstawiono rezultaty działania algorytmów SocialPageRank, AdaptedPageRank. Wyniki te zostały posortowane w zależności od wyniku algorytmu SocialPageRank \ref{tab:social_page_wyniki_sorted} i  AdaptedPageRank \ref{tab:adapted_wyniki_sorted}. Dodatkowo w tablicy \ref{tab:inne_wyniki_sorted} znajdują się wyniki posortowane w zależności od sumy popularności dokumentów w sieciach twitter, facebook i digg. Można zauważyć, że żaden z pierwszych 25 dokumentów się nie pokrywa. 

W tablicy \ref{tab:rozne_wyniki}  znajdują się wartości maksymalne, minimalne i średnie dla różnych algorytmów. Można zauważyć, że podobnie jak dla prostych przykładów, algorytm AdaptedPageRank ma zdecydowanie mniejsze różnice miedzy poszczególnymi wynikami. 


\begin{table}[htb]
  \centering
    \begin{tabular}{ | l | l | l | l| }
\hline
  & maksymalna & średnia & minimalna \\
\hline
SocialPageRank    &  59.9655565  & 0.654392541 &  1.4996654e-012 \\
AdaptedPageRank &   0.7196510  & 0.497870340 &  0.2520965 \\
inne            & 578287638         & 17132.4861 (21544.9729)   & 0 \\
\hline
\end{tabular}
  \caption{ Maksymalne, średnie, minimalne wartości algorytmów. }
  \label{tab:rozne_wyniki}
\end{table}


Dokładniej zelażnością między poszczególnymi wynikami można się przyjrzeć na rysunkach \ref{fig:social-adapted}, \ref{fig:adapted-social}, \ref{fig:adapted-inne}, \ref{fig:social-inne}. Na wykresach tych przedstawiono wyniki algorytmów posortowane po jednej z danych. Dodatkowo na rysunku \ref{fig:adapted-inne} i \ref{fig:social-inne} znajdują się wyniki odpowiednio algorytmu AdaptedPageRank i SocialPageRank przedstawione razem z danymi uzyskanymi z innych sieci społecznościowych. Na \ref{fig:social-inne} można zaobserwować zależność między wynikami algorytmu SocialPageRank a popularnością danych dokumentów w innych sieciach społecznościowych.



\begin{figure}[htbp]

    \centering
    \includegraphics[width=\linewidth]{social-adapted.png}
    \caption{Zależności między wynikami algorytmów SocialPageRank i AdaptedPageRank. Posortowane po wartościach algorytmu SocialPageRank}
    \label{fig:social-adapted}
\end{figure}

\begin{figure}[htbp]
    \centering
    \includegraphics[width=\linewidth]{adapted-social.png}
    \caption{Zależności między wynikami algorytmów SocialPageRank i AdaptedPageRank. Posortowane po wartościach algorytmu AdaptedPageRank}
    \label{fig:adapted-social}

\end{figure}



\begin{figure}[htbp]

    \centering
    \includegraphics[width=\linewidth]{adaoted-inne.png}
    \caption{Zależności między wynikami algorytmów AdaptedPageRank i danymi z innych sieci socialnych. Posortowane po wartościach z innych sieci}
    \label{fig:adapted-inne}
\end{figure}

\begin{figure}[htbp]
    \centering
    \includegraphics[trim = 0mm 10mm 0mm 0mm, width=\linewidth]{social-inne.png}
    \caption{Zależności między wynikami algorytmów SocialPageRank i danymi z innych sieci socialnych. Posortowane po wartościach z innych sieci}
    \label{fig:social-inne}

\end{figure}
















