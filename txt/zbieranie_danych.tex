\section{Zbieranie danych}

	[TODO - OPIS CRAWLERÓW]

\subsection{Czyszczenie tagów przed zapisem}

Tagi pobierane z serwisu delicous nie zawsze są w postaci wymaganej przez aplikacje. Spowodowane to jest błędami użytkowników, czy też specyficznym stylem zapisywania tagów.

Niektóre tagi mają różne znaczenie w zależności od kontekstu, na przykład tag 'design' ma inne znaczenie w kontekście strony o programowaniu, a inne w kontekście strony o sztuce. Część użytkowników żeby poradzić sobie  z tym problemem dodają do tagów informacje mówiące o ich domenie. Często domena ma wygląd 'programming@design' czy 'art\#design'. 

Z powodu tego specyficznego zapisu każda etykieta, przed zapisaniem do bazy danych jest dzielona na 2 lub więcej słów w miejscach występowania popularnych znaków specjalnych. Dlatego też 

Dodane kontekstu do tagu mogłoby być przydatne w aplikacji, ale z powodu tego że każdy użytkownik ma swój specyficzny sposób opisywania dokumentów np: design@art i art\#design, trudno jest je zunifikować. Dodatkowym problemem jest to, że nie jest to sposób opisu używany przez wszystkich użytkowników. 


Adnotacje przypisywane przez użytkowników często kończą się lub zaczynają od znaków specjalnych. Jest to spowodowane np: błędami (dodatkowe przecinki) albo specyficznym stylem opisywania danych przez użytkownika. Wszystkie znaki specjalne z końca i początku dokumentu są usuwane przed dodaniem do bazy danych.

Przykłady danych przed i po ich oczyszczeniu:

\begin{itemize} 
    \item '@java' : 'java'
    \item '@@java' : 'java'
    \item  '\#java6@' : 'java6'
    \item  'design!\$\%@art' : 'design', 'art'
    \item  'art!\#,': 'art'
\end{itemize}