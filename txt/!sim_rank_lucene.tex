\section{Lucene}


Niech dla danego termu t, dokumentu (lub zapytania) x.


Wartość dokumentu $d$ w przestrzeni wektorowej dla zapytania $q$ wyliczana jest z podobieństwa cosinusów wektorów $V(d)$ i $V(q)$:

\begin{equation}
cosine\_similarity(q,d) = \frac{V(q) \cdot V(d) }{ |V(q)||V(d)|}
\end{equation}

Gdzie $V(q) \cdot V(d)$ jest iloczynem skalarnym ważonych wektorów $V(q)$ i $V(d)$, a $|V(q)|$ i $|V(d)|$ są ich normami euklidesowymi. 

We frameworku lucene wartość tej funkcji została zmieniona. Powodem tego była użyteczność i jakoś wyników. 

\begin{itemize}
\item wartość |V(d)| 
\end{itemize}


\begin{equation}
score(q,d) =   coord(q,d)    queryNorm(q) 
\sum_{t \text{ in } q}  \bigg( tf(t\text{ in } d)    idf(t)^2    getBoost(t)   norm(t,d) \bigg)
\end{equation}

gdzie:
\begin{itemize}
	\item coord(q,d): funkcja zwraca wartości zależne od miejsca występowania i odległości od siebie szukanych termów w dokumencie.
	\item queryNorm(q): funkcja normalizująca wyniki zapytania
	\item tf(t in d): funkcja wyliczająca częstość występowania danego termu w dokumencie
	\item idf(t) : funkcja wyliczająca częstość występowania termu we wszystkich dokumentach.
	\item getBoost(t) - Lucene pozwala na zwiększenie wagi niektórych termów. Nieużywane w tej aplikacji.
\end{itemize}

Lucene ocenia dokumenty głównie na podstawie funkcji TF-IDF. Każdy dokument reprezentowany jest przez wektor, składający się z wag słów występujących w tym dokumencie. TF-IDF informuje o częstości wystąpienia termów uwzględniając jednocześnie odpowiednie wyważenie znaczenia lokalnego termu i jego znaczenia w kontekście pełnej kolekcji dokumentów.


\subsection{Pobieranie stron}
