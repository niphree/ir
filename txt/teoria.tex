\chapter{Opis algorytmów}


\section{Folksonomia}

Folksonomia nazywamy społeczne klasyfikowanie czy tagowanie różnych zasobów. Formalnie model folksonomii został przedstawiony w pracy \cite{hotho2006information} i zdefiniowany został jako: 

\begin{definicja}
Folksonomia jest to krotka $F := (U,T,R,Y, )$, gdzie:
\begin{itemize}
\item $U$,$T$ i $R$ są to zbiory skończone, których elementy nazywają się odpowiednio użytkownicy, tagi i dokumenty
\item $Y$ jest relacją między tymi elementami: $Y \subseteq  U \times T \times T $, nazywamy ją relacją przypisania tagu.

\end{itemize}
\end{definicja}

Użytkownicy i tagi identyfikowani są na podstawie ich unikalnych nazw własnych i identyfikatorów. Dokumenty mogą być różnymi danymi: stronami www, zdjęciami, plikami np: pliki pdf. Ta praca bazuje na danych pobranych z witryny delicous, które w zdecydowanej większości są stronami www. Dane, które nie są stroną www nie są brane pod uwagę w tej pracy. 

\section{Opis metod rankingu dokumentów}

Do wyliczenia rankingu dokumentów zostały użyte kilka metod. 

AdaptedPageRank i SocialPageRank, są to algorytmy którę biorą pod uwagę popularność tagów, użytkowników, dokumentów i relację między nimi. 

Kolejnym użytym w pracy rankingiem użytym w pracy jest wersja algorytmu tf-idf użyta we frameworku Lucene. 

Kolejny ranking stanowią dane z serwisów takich jak facebook, twitter czy digg mówiące o ilości udostępnień danego zasobu przez użytkowników tych serwisów. 








