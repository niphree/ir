\chapter{Opis algorytmów}


\section{Folksonomia}



Użytkownicy i tagi identyfikowani są na podstawie ich unikalnych nazw własnych i identyfikatorów. Dokumenty mogą być różnymi danymi: stronami www, zdjęciami, plikami np: pliki pdf. Ta praca bazuje na danych pobranych z witryny delicous, które są stronami internetowymi. Dane, które nie spełniają tego warunku, nie są brane pod uwagę w tej pracy. 

\section{Opis metod rankingu dokumentów}

Do wyliczenia rankingu dokumentów zostało użyte kilka metod. Pierwsze dwie metody to statyczne rankingi Adapted PageRank i SocialPageRank. Algorytmy te biorą pod uwagę popularność tagów, użytkowników, dokumentów i wzajemne relację między nimi. 

Kolejnym użytym w pracy rankingiem jest wersja algorytmu TF-IDF zaimplementowana we frameworku Lucene. 

Ostatnią metodą stanowiącą o popularności dokumentów są dane z serwisów takich jak facebook, twitter czy digg mówiące o ilości udostępnień danego zasobu przez użytkowników tych serwisów. 








