\chapter{Opis problemu}

Praca ma na celu zbadanie problemu wyliczania rankingu dokumentów w sieciach społecznościowych w czasie wyszukiwania wyników zapytania. Badanie w tej pracy sieci społecznościowe charakteryzują się tym,że zasoby przechowywane przez nie zostały przez różnych użytkowników wybranymi etykietami (zwanymi również annotacjami czy tagami). Jako dodatkowy element, w czasie wyszukiwania dokumentów zostaną użyte dane mówiące o popularności danych wyników, pozyskane z rożnych serwisów społecznościowych, takich jak twitter, facebook czy digg. Dane te powinny polepszyć jakoś wyszukiwania i przyśpieszyć obliczenia potrzebne do wyliczenia statycznych rankingów dokumentów. 