\chapter{Algorytmy}


\section{Folksonomia}

Folksonomią nazywamy społeczne klasyfikowanie, czy tagowanie różnych zasobów. Formalnie model folksonomii został przedstawiony w pracy \cite{hotho2006information}. Można ja przedstawić jako krotkę $F := (U,T,R,Y)$, gdzie:
$U$,$T$,$R$ to zbiory skończone, których elementy składają się odpowiednio z użytkowników, tagów i dokumentów. $Y$ jest relacją “przypisania tagu” pomiędzy tymi elementami $Y \subseteq U \times T \times R$

Użytkownicy i tagi identyfikowani są na podstawie ich unikalnych nazw własnych. Dokumenty mogą być różnymi danymi: stronami www, zdjęciami, plikami np: pliki pdf. Ta praca bazuje na danych pobranych z witryny delicous, które w zdecydowanej większości są stronami www. Dane które nie są stroną www nie są brane pod uwagę w tej pracy. 


