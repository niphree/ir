\section{Opis Aplikacji}

Aplikację można podzielić na 3 główne części:

\begin{itemize}
\item interfejs użytkownika pozwalający na wyszukiwanie w zasobach i wyświetlający wyniki użytkownikowi.
\item Główną część aplikacji, odpowiedzialna za preprocessing i wyliczanie algorytmów.
\item Crawlery zbierające nowe dane i odświeżające już posiadane.
\end{itemize}

W pierwszej części odbywa się wykonywanie zapytań, wyliczenie ostatecznego rankingu dokumentów i wyświetlanie odpowiedzi użytkownikowi. Wykorzystany jest tutaj framework Lucene i wcześniejsze obliczenia wykonane w innych fragmentach aplikacji. 

Druga część aplikacji zawiera głównie implementacje algorytmów Social PageRank, Adpated PageRank jak również odpowiada za wykonywanie wcześniejszych obliczeń (preprocessing), które przyśpieszają późniejsze wykonanie algorytmów. 


Ostatnia część składa się z różnych wątków spełniających rolę crawlerów. Kilka z tych wątków pobiera i odświeża zapisane poprzednio dane z serwisu delicous. Inne odpowiadają za pobranie informacji o popularności danych zasobów w serwisach twitter, digg i facebook. Ostatni z crawlerów pobiera dla każdego dokumentu jego treść i zapisuje go przy użyciu frameworku Lucene na dysk. Dokładniejszy opis zbierania danych można znaleźć w rozdziale \ref{sec:pobieranie_danych}

Wszystkie wymienione poprzednio fragmenty aplikacji korzystają z bazy danych, która bardziej szczegółowo zostanie opisana w kolejnych rozdziałach.