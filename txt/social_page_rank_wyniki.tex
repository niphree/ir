\section{SocialPageRank: wyniki}

\subsection{Wyniki algorytmu dla prostych danych}

W tabelce \ref{fig:test_proste_dane} znajdują się dane, dla których zostało sprawdzone działanie algorytmu Social PageRank. Dane są nie duże i składają się z trzech różnych dokumentów, dwóch użytkowników i trzech tagów.

\begin{figure}
  \centering
  \begin{tabular}{|c|c|c|c|c|}
    \hline
    \multicolumn{1}{|c|}{}&\multicolumn{2}{c|}{Użytkownicy}\\

    \cline{2-3}
    \multicolumn{1}{|c|}{}&użytkownik 1&użytkownik 2\\
    \hline
 	http://www.ted.com/ & inspiration & \\
	http://www.colourlovers.com/&	design & inspiration \\
	http://www.behance.net/	&portfolio, design & portfolio, inspiration \\
    \hline
  \end{tabular}
  \caption{Dane do prostych testów}
  \label{fig:test_proste_dane}
\end{figure}

Dla takich danych macierz $M_{d,u}$ mówiąca o zależności dokumentów z użytkownikami ma postać:

\[
 M_{d,u} =
 \begin{pmatrix}
  1 & 0 \\
  1 & 1 \\
  1 & 2
 \end{pmatrix}
\]

Macierz użytkowników i tagów, $M_{u,t}$:

\[
 M_{u,t} =
 \begin{pmatrix}
  1 & 1 & 1 \\
  2 & 0 & 1 
 \end{pmatrix}
\]

Macierz tagów i dokumentów, $M_{t,d}$:
\[
 M_{t,d} =
 \begin{pmatrix}
  1 & 1 & 1 \\
  0 & 1 & 1 \\
  0 & 0 & 2 
 \end{pmatrix}
\]


\subsection*{wyniki:}
Dla powyższych danych wyniki algorytmu zbiegają po czterech iteracjach z dokładnością
$|P_3 - P_4|  < 10^{-10}$. Wyniki zostały przedstawione w tabelce \ref{fig:social_page_simple_wyniki}


\begin{table}[h]
  \centering
    \begin{tabular}{ | c | c | }
\hline
&Social PageRank \\
\hline
www.ted.com & 0.2381373691295440 \\
www.colourlovers.com  & 0.4343479235414989 \\
www.behance.net & 0.8686958470829979 \\
\hline
\end{tabular}
  \caption{Wyniki działania algorytmu Social PageRank}
  \label{fig:social_page_simple_wyniki}
\end{table}


Można zauważyć, że największy ranking ma strona behance.net, która została dodana przez dwóch użytkowników i oznaczonych najpopularniejszymi tagami - 2 razy tagiem portfolio, użytym tylko dla tej strony, raz tagiem design, który użyty był 2 razy w powyższych danych i również raz tagiem inspiration, który jest najpopularniejszym tagiem, użytym w przykładzie aż 3 razy. 











