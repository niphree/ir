\section{Adapted PageRank: wyniki}

\subsection{Wyniki algorytmu dla prostych danych}


Algorytm Adapted PageRank przetestowano na tych samych danych co algorytm Social PageRank (tabelka \ref{fig:test_proste_dane}). Macierz asocjacyjna powstała z tych danych ma wymiary $8 \times 8$ i wygląd:

\[
 G_f =
 \begin{pmatrix}
0 & 0 & 0 & 1	 & 0 & 1 & 0 & 0\\
0 & 0 & 0 & 1 & 1 & 1 & 1 & 0\\
0 & 0 & 0 & 2 & 2 & 1 & 1 & 2\\
1 & 1 & 2 & 0 & 0 & 1 & 2 & 1\\
0 & 1 & 2 & 0 & 0 & 2 & 0 & 1\\
1 & 1 & 1 & 1 & 2 & 0 & 0 & 0\\
0 & 1 & 1 & 2 & 0 & 0 & 0 & 0\\
0 & 0 & 2 & 1 & 1 & 0 & 0 & 0
 \end{pmatrix}
\]

\begin{table}[h]
  \centering
    \begin{tabular}{ | c | c | }
\hline
&Adapted PageRank \\
\hline
doc: http://www.ted.com/ & 0.280676409730572 \\
doc: http://www.colourlovers.com/ & 0.369220441236174 \\
doc: http://www.behance.net/ & 0.384551423972747 \\
\hline
usr: użytkownik A & 0.402473669662513 \\
usr: użytkownik B & 0.354578057974890 \\
\hline
tag: inspiration & 0.383291185401107 \\
tag: design & 0.321455285546253 \\ 
tag: portfolio	 & 0.314739469615726 \\
\hline
\end{tabular}
  \caption{Wyniki działania algorytmu Adapted PageRank}
  \label{fig:adapted_page_simple_wyniki}
\end{table}

Zbierzność wektora została uzyskana po 22 iteracjach. Jest to zdecydowanie dłuższy czas w porównaniu z czterema wymaganymi iteracjami przy poprzenim algorytmie


Analizując wyniki z tabeli \ref{fig:test_proste_dane} można zauważyć, że najwyższy ranking wśród dokumentów ma strona begence.net: została ona dodana przez 2 użytkowników i przypisane jej zostały 4 tagi. Niewiele niższy ranking ma witryna colourlovers dodana przez 2 uzytkowników i opisana 2 różnymi annotacjami. Można po tym wywnioskować ze nadanie większej ilości tagów nie ma dużego wpływu na rank strony. Za to zmniejszenie liczby użytkowników którzy tą strone dodali, ma duze: przykład strona ted.com i colourlovers.com gdzie widoczny jest dość duzy skok wartości wyniku.



