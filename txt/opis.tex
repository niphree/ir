\documentclass[11pt,a4paper]{report}

\usepackage[polish]{babel}
\usepackage[utf8x]{inputenc}
\usepackage{polski}
%\usepackage[T1]{fontenc}
\frenchspacing
\usepackage{indentfirst}
\usepackage{amsthm}
\usepackage{amsmath}
\usepackage{algorithmic}
\usepackage{algorithm}
\usepackage{program}
\usepackage{programs}
\usepackage{array}
\usepackage{multirow}
\usepackage{graphicx}
\usepackage{listings}
\usepackage{listing}
\usepackage{float}
\pagenumbering{arabic}

\graphicspath{{img/}}

\begin{document}
\tableofcontents
\chapter{Opis}

W czasie tworzenia aplikacji natknełam się na kilka problemów. Podstawowym z nich była ilość danych, na których aplikacja operowała. 

Algorytmy zaimplementowane operują na macierzach. Algorytmy te są odmianą algorytmu PageRank. Głównymi operacjami przeprowadzanymi w tym algorytmach są operacje macierz-wektor. Gdzie główną rzeczą odróżniającą te algorytmy jest sposób tworzenia macierzy. 

while not(zbierznosc wektora v) do
	\# wykonaj operacje macierz-wektor i przypisz do wektora v
	v := do\_some\_calculation(M, v)

Macierz używana w każdej iteracji jest niezmienna.

Idealnie byłoby trzymać macierz cały czas w pamięci. Ale niestety wielkość macierzy nie pozwala na takie operacje. 


\begin{itemize}
\item Wyszukiwanie w tekście. - użycie lucene (DOPISAC)

\end{itemize}

\end{document}