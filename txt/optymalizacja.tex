\chapter{Implementacja}

\Section{Baza danych}

opis bazy danych

Jako serwer bazy danych został użyty MySql serwer z silnikiem bazy danych InnoDB. Innymi rozważanymi 

Baza danych składa się z trzech głównych tabel: Users, Documents i Tags. Zawierają one informację na temat użytkowników, dokumentów i adnotacji pobrane z serwisu delicous. W bazie danych znajdują sie też tabele: UserTagDoc i UserTagDoc_tag, które służą do zapisania relacji między użytkownikami a dokumentami (usertagdoc) i adnotacjami (UserTagDoc_tag). W tych tabelach zapisane są  informacje na temat tego czy dany użytkownik dodał dokument do serwisu i jakimi tagami została dana strona opisana.

Schemat bazy danych:

Create table


Dodatkowo w bazie danych znajdują się dwie tabele: tag_doc i tag_user. Tabele tych zapisywane są dane wyliczone z pozostałych tabel. W tabeli tag_doc znajdują się informację na temat tego ile razy przez różnych użytkowników dany dokument doc_i  został dodany i opisany tagiem tag_k. Odpowiednia w tabeli tag_user znajdują się informację na temat ilości różnych dokumentów dodanych przez użytkownika usr_n i opisanych tagiem tag_m. Dane ilości różnych tagów którymi użytkownik usr_l opisał dokument doc_j przechowywane są w już istniejącej tabeli UserTagDoc.

Schemat bazy danych z zaznaczonymi nowymi tabelami i polem count w usertagdoc

Dla przyśpieszenia działania bazy

